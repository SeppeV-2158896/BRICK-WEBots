\documentclass[a4paper]{article}

%% Language and font encodings
\usepackage[english]{babel}
\usepackage[utf8x]{inputenc}

\usepackage{booktabs}
\usepackage{tabu}
\usepackage[T1]{fontenc}

%% Sets page size and margins
\usepackage[a4paper,top=3cm,bottom=2cm,left=3cm,right=3cm,marginparwidth=1.75cm]{geometry}

%% Useful packages
\usepackage{amsmath}
\usepackage{graphicx}
%\usepackage{apacite}
\usepackage[colorinlistoftodos]{todonotes}
\usepackage[colorlinks=true, allcolors=blue]{hyperref}

\title{Software Engineering for Cyberphysical Devices: \\ Design Sheet}
\author{Pex Jan-Mathijs, Thys Ferre and Vandenberk Seppe}
\date{}

\begin{document}
\maketitle

\section*{System Requirements}

The system needs to be capable of moving around an industrial facility to survey the area and measure various data elements. This means that the device should be able to move under user control as well as being directed to specific points within a room. The device doesn't require the ability to carry more than its own weight. Power for the device should come from the robot itself, necessitating a battery-operated solution without the need for cables. The company intends to control the robot through a web browser, meaning its operational range is constrained by network coverage.

The industrial facility may feature multiple floor types, including the possibility of wet and dusty floors, as the robot's primary purpose is to detect errors. Although the rooms won't have stairs, the potential for ramps exists.

Ultimately, the robot should be capable of performing inspection, repair, and maintenance tasks.

\section*{Mechanical Design}

\begin{itemize}
\item Tracks are more versatile and stable then wheels. They also provide more traction across multiple floor types. Possible addition of weight won't be much of a problem, because of the weight distribution and the higher payload capacity compared to regular wheels. Finally, wheeled robots may have more maneuverbility in tight spaces, tracks can still navigate efficiently enough through industrial environments. 
\item The idea is to build a low profile bot, with a telescopic arm, so it fits better in to tight spaces. The use of a skid plate will prevent the spraying water from entering the system.
\end{itemize}

\section*{Electrical Design}

\begin{itemize}
    \item \textbf{Raspberry Pi}: The Raspberry Pi serves as the brain of the robot, controlling its behavior and components. It processes sensor data, executes algorithms, and sends data over the cloud to our interfacing software.
    
    \item \textbf{Sensors}: Our robot is equipped with an array of sensors for perception and environmental awareness. These sensors include:
    \begin{itemize}
        \item \textbf{Distance Awareness}: Our system will use a Lidar sensor for distance perception. This in combination with a GPS sensor, can help determine the robots location.
        \item \textbf{Inertial Measurement Unit (IMU)}: Provides data on orientation, acceleration, and angular velocity for navigation and motion control.
        \item \textbf{Camera(s)}: Enables visual perception and object recognition for tasks such as navigation and object manipulation.
        \item \textbf{Environmental Sensors}: Measure parameters such as temperature, humidity, and atmospheric pressure for environmental monitoring and adaptation.
        \item \textbf{Safety Sensors}: A safety sensure, such as a touch sensor can prevent the robot from doing unwanted behavior. 
    \end{itemize}
    
    \item \textbf{Actuators}: 
    \begin{itemize}
        \item \textbf{DC Motors}: Used for driving mechanisms.
        \item \textbf{Servo Motors}: Provide precise control over angular position for tasks such as camera orientation.
    \end{itemize}
    
    \item \textbf{Power Supply}: The power supply system based on batteries, provides electrical energy to the robot's components. It includes voltage regulators and power distribution circuits to ensure stable and reliable operation.
    
\end{itemize}

\section*{Software Design}
4.	Software:
o	Python woooooo
o	Specify the control algorithms for motion planning, collision avoidance, and task execution.
1.	D* lite
2.	Lidar with threshold for collision avoidance, bumper with pressure sensor as an extra safety measure in case the lidar mechanism fails.
o	Include safety features and emergency stop procedures.

\section*{Safety Systems}
Our robot will have safety systems, so when it bumps into objects or enters restricted areas the robot won't be allowed to move any further, and will require to drive back in the distance it came from to avoid more damage. This will also require a re-calculation of the path it needs to take. The addition of a emergency stop in the web-interface and a physical button on the robot, will allow human interference to stop the robot at any time.

\section*{Human-Machine Interaction (HMI)}
6.	Human-Machine Interface (HMI):
o	Describe the user interface for programming, monitoring, and troubleshooting.
1. Web interface, robot is keyboard controlled (with ps4 option ofc) , 		visible map on which to set goal points (points to navigate to)
Also some other controls and sensor interface. Live video feed of 		camera via WebSockets 	

o	Specify the level of automation and manual control options.
2. Given a point on the map the bot should be able to in a safe and pid controlled way find its way to the point without extensive stopping, preferrably in one smooth motion
o	Include error messages and diagnostics.

\section*{Environmental Considerations}
To develop a good simulation of the system, it is crucial to asses the environmental factors that could influence the efficiency and durability of the robot. These factors encompass a range of elements, including terrain, weather conditions, power requirements, and sustainability considerations.

When it comes to terrain, it is essential to make some assumptions about the surrounding of the robot. Whether it will be navigating indoor spaces with smooth surfaces or outdoor terrains with rough surfaces and obstacles, the selection of mobility mechanisms like wheels or tracks becomes vital. For this design, the robot will be limited to an area that is covered by a wireless network. So the robot will be designed for indoor use, but the floors can be wet and dusty due to it being an industrial facility and the main task of the robot to surveil and detect any problems.

Weather conditions don't significantly impact the operational efficiency and longevity of the robot. Because of its indoor use. Therefore, there will be no direct action taken to prevent wheather damage to the robot. The robot we design will be splash proof, to prevent spraying water from the tracks from entering the system. Since it is an indoor industrial facility, the assumption is made that the temperature will not exceed the working temperatures of the equiment

Power considerations are fundamental in determining the autonomy and dependability of the robot. Because of the intended use the robot should be able to function on a battery and will be able to charge itself in the facility. Most of todays industrial facility are not to big to drive around in one singular battery charge, so this is also an assumption we take for this project.

\section*{Maintenance}
o	Outline maintenance procedures, including lubrication and inspection schedules.
o	Specify spare parts availability and ease of replacement.
o	Include documentation for troubleshooting and repairs. 



\bibliographystyle{plain}
\bibliography{refs}

\end{document}